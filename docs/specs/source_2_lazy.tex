\input source_header.tex

\begin{document}
	%%%%%%%%%%%%%%%%%%%%%%%%%%%%%%%%%%%%%%%%%%%%%%%
	\docheader{2021}{Source}{\S 2 Lazy}{Jellouli Ahmed,
        Ian Kendall Duncan,
        Cruz Jomari Evangelista,
        Martin Henz, Alden Tan}
	%%%%%%%%%%%%%%%%%%%%%%%%%%%%%%%%%%%%%%%%%%%%%%%

\input source_intro.tex

Source \S 2 Lazy is a lazy-evaluation variant of Source \S 2.

\section*{Changes}

Source \S 2 Lazy modifies Source \S 2 by using
lazy evaluation. In
this scheme, the argument expressions of functions are passed un-evaluated
to the function to which they are applied. The function then evaluates
these expressions whenever their values are required. This evaluation
method is generally
called normal order reduction. If functions
do not have any side-effects, as in Source \S 2,
there is no need to evaluate such an expression
multiple times, as the result is guaranteed to be the same. This observation
leads to the variant of normal order reduction, called \emph{lazy evaluation}.
In the lazy evaluation language Source \S 2 Lazy,
the evaluator remembers the result of evaluating the
argument expressions for the first time, and simply retrieves this result
whenever it is required again.

Lazy evaluation is used for all arguments of user-defined functions and
for the primitive functions \lstinline{pair}, \lstinline{head} and
\lstinline{tail}, but
not for any arguments of other primitive functions or operators.

\newpage

\input source_bnf.tex

\input source_2_bnf.tex

\input source_boolean_operators

\input source_return

\input source_names

\input source_lists

\input source_numbers

\input source_strings

\input source_typing

\input source_comments

\input source_js_differences

\newpage

\input source_list_library

    \end{document}
