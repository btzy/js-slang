\input source_header.tex

\begin{document}
	%%%%%%%%%%%%%%%%%%%%%%%%%%%%%%%%%%%%%%%%%%%%%%%
	\docheader{2021}{Source}{\S 4 GPU}{Martin Henz, Rahul Rajesh, Zhang Yuntong}
	%%%%%%%%%%%%%%%%%%%%%%%%%%%%%%%%%%%%%%%%%%%%%%%

\input source_intro.tex

\section{Changes}

Source \S 4 GPU allows for Source programs to be accelerated on the GPU if certain conditions are met.
The exact specifications for this is outlined on page \pageref{gpu_supp}. Source \S 4 GPU  defines a formal specification 
to identify areas in the program that are embarrssingly parallel (e.g. for loops etc.) . These will then
be run in parallel across GPU threads. Experimentation has shown that Source \S 4 GPU is orders of magnitude faster
than Source \S 4 for heavy CPU bound tasks (matrix multiplication of large matrices)

\input source_bnf.tex

\input source_3_bnf.tex

\newpage

\input source_return

\input source_import

\input source_boolean_operators

\input source_loops

\input source_names_lang

\input source_numbers

\input source_strings

\input source_arrays

\input source_comments

\input source_typing_3

\section{Standard Libraries}

The following libraries are always available in this language.

\input source_misc

\input source_math

\input source_lists

\input source_pair_mutators

\input source_array_support

\input source_streams

\input source_interpreter

\input source_js_differences

\newpage

\section*{GPU Acceleration}
\label{gpu_supp}
This section outlines the specifications for programs to be accelerated using the GPU.\
\input source_gpu_bnf.tex

\newpage

\section*{Restrictions}

Even if the BNF syntax is met, GPU acceleration can only take place if all the restrictions below are satisfied. If all criteria are met, the \textit{gpu\_statement} loops are embarrassingly parallel.

\subsection*{Special For Loops}

In the BNF, we have special loops that take on this form:
\begin{alignat*}{9}
&& \textbf{\texttt{for}}\ \textbf{\texttt{(}} 
                          \ \textit{gpu\_for\_let} \textbf{\texttt{;}} \\
&& \ \ \textit{gpu\_condition} \textbf{\texttt{;}} \\
&& \textit{gpu\_for\_assignment} \ \textbf{\texttt{)}} 
\end{alignat*}

These are the loops that will be taken into consideration for parallelization. However, on top of the BNF syntax, the below requirement must also be statisfied:

\begin{itemize}
    \item{the names declared in each \textit{gpu\_for\_let} have to be different across the loops}
    \item{in each loop, the \textit{gpu\_condition} and the \textit{gpu\_for\_assignment} must use the name declared
    in the respective \textit{gpu\_for\_let} statement}
\end{itemize}

\subsection*{GPU Function}

A \textit{gpu\_function} has to be a \textit{math\_\texttt{*}} function

\subsection*{Core Statement}

Within \textit{core\_statement}, there are some constraints:

\begin{itemize}
    \item{no assignment to any global variables (all assignments can only be done to variables defined in the \textit{gpu\_block}})
    \item{no use of the variable in \textit{gpu\_result\_assignment} at an offset from the current index e.g. cannot be i - 1}
\end{itemize}

\subsection*{GPU Result Statement}

The \textit{gpu\_result\_assignment} is the statement that stores a value calculated in core statements into a result array. 
It access an array at a certain coordinate e.g. ${array[{i_1}][{i_2}][{i_3}]}$. For this:

\begin{itemize}
    \item{This result array has to be defined outside the \textit{gpu\_block}.}
    \item{The sequence of coordinates which we access in the result array ${{i_1}, {i_2}, {i_3} ... i_{k}}$ must be a 
        prefix of the special for loop counters ${[c_1,c_2 ... c_n]}$.}
    \item{ If you have ${n}$ special for loops, the array expression can take on ${k}$ coordinates where ${0 < k \leq n}$. 
    The order matters as well, it has to follow the same order as the special for loops: you cannot have ${name[c_2][c_1]}$.}
\end{itemize}

\section*{Examples}

Below are some examples of valid and invalid source gpu programs:\\

\textbf{Valid} - Using first loop counter. (meaning the loop will be run across N threads; the first loop is parallelized away):
\begin{verbatim}
for (let i = 0; i < N; i = i + 1) {
    for (let k = 0; k < M; k = k + 1) {
        res[i] = arr[k % 2] + 1;
    }
}
\end{verbatim}

\textbf{Invalid} - Counter used is not a prefix of for loop counters:
\begin{verbatim}
for (let i = 0; i < N; i = i + 1) {
    for (let k = 0; k < M; k = k + 1) {
        res[k] = arr[i % 2] + 1;
    }
}
\end{verbatim}

\textbf{Valid} - Using first three loop counters (meaning the loop will be run across N*M*C threads, if available):
\begin{verbatim}
for (let i = 0; i < N; i = i + 1) {
    for (let j = 0; j < M; j = j + 1) {
        for (let k = 0; k < C; k = k + 1) {
            let x = math_pow(2, 10);
            let y = x * (1000);
            arr[i][j][k] = (x + y * 2);
        }
    }
}
\end{verbatim}

\textbf{Invalid} - Indices are in wrong order (must respect for loop counter orders):
\begin{verbatim}
for (let i = 0; i < N; i = i + 1) {
    for (let j = 0; j < M; j = j + 1) {
        for (let k = 0; k < C; k = k + 1) {
            let x = math_pow(2, 10);
            let y = x * (1000);
            res[k][j][i] = (x + y * 2);
        }
    }
}
\end{verbatim}

\textbf{Invalid} - Using an index that is not part of a special for loop (see above):
\begin{verbatim}
for (let i = 0; i < N; i = i + 1) {
    for (let j = 0; j < M; j = j + 1) {
        for (let k = 1; k < C; k = k + 2) {
            res[k] = arr1[i] + arr2[j];
        }
    }
}
\end{verbatim}

\newpage

\input source_list_library

\newpage

\input source_stream_library


\end{document}
